\documentclass{article}
\usepackage{fullpage}
\begin{document}
\section*{Modeling the gravity darkening signal of the giant planets in Kepler}
The angle between a star's rotational axis and the normal to the orbital plane of the system's planets may provide clues about the history of the system.
Spin orbit alignment has been suggested to be evidence for disk migration (Lin et al, 1996), while spin orbit misalignment has been suggested as evidence for dynamic migration.
The main motivation for this project is to examine the population of planets around fast rotating stars, and their spin obliquities.

Methods for measuring spin obliquity include doppler tomography, $V\sin i$, astroseismology, and the Rossiter-McLaughlin effect.
The common method for measuring spin obliquity uses spectroscopic analysis to measure the Rossiter-McLaughlin effect.
While the Rossiter-McLaughlin effect has been used on many systems, but isn't useful in all cases.
For rapidly rotating stars, the spectral lines become broad and the spin obliquity angle can't be reliability measured.
In this case, a photometric model can be used.


The gravity darkening signal is a unique method for measuring the true spin obliquity angle of a planetary system.
Rapid rotation causes a star to become oblate, and causes the star's poles to become hotter, and the equator to become cooler.
This is known as the gravity darkening effect (von Zeipel,1924).
A latitudinal differential in the of the stellar surface causes misaligned planets to produce asymmetric light curves.
Gravity darkening is useful because, unlike other methods, it can constrain both the stellar obliquity and the sky-projected spin-orbit alignment.
It is also useful for measuring spin obliquity angles for hot stars, which is currently an area lacking good measurements.


While a photometric model may provide useful information for the features of a system, in practice this technique has provided varying results between teams and contradicts  results from doppler tomography ( in the case of KOI-13).
Current methods use the default Van Zipel model and perform fits with linear limb darkening models.
These methods can be improved with a two parameter limb darkening model, an updated gravity darkening model, and a modern fitting method, such as a Monte-Carlo Markov chain.


In this project, we propose to improve the current methods of gravity darkening modelling, and provide self consistent modeling of the giant planet and super earth systems in the Kepler data.
We will implement better models to describe gravity and limb darkening, and attempt to understand model bias and completeness in contrast to previous works.
This will provide the first population of self-consistent measured spin obliquity angles for planets around hot stars.


The project will start with the gravity darkening models used by Zhou \& Huang (2013), and make several critical improvements.
Firstly, Zhou and \& Huang (2013) only made corrections from the gravity darkening effect to the regular Mandel \& Agol transit model, without the corrections for oblateness.
Another, more advanced gravity darkening model implemented in Brandt \& Huang 2015a will also be used.
This model is a better description of the observed surface of rapid rotating stars compared to the traditional Von Zipel model.
In addition, we will also implement different limb-darkening laws and assess the impact of them to our final result.

To determine the accuracy of these models, they will be applied to simulated light curves.
This provides a controlled manner to test the models and determine what information can be recovered.
This process sets a baseline and expectation for modeling real data.
After testing the models on simulated data, the models will be applied to Kepler systems of interest.
Modeling Kepler systems that have been extensively studied, such as Kepler 13, will provide a gauge for our modeling technique versus numerous results with other methods in both photometric and doppler based observations.


The goal with the development process of this project is to create a pipeline useful for both this project, but also for future transit modeling effort.
The packages available for modeling exoplanet transits lack the flexibility needed to model certain questions.
The pipeline we create will be an excellent byproduct for this project.
The models themselves will be created in C and implemented into a python package with Swig.
This provides an easy to use and maintain code base, without sacrificing efficiency.
The fitting process will use the EMCEE package to perform a Monte-Carlo Markov chain(MCMC).
The difficulty in developing this pipeline will be creating a model that can run efficiently enough to be used with an MCMC.
\section{Notes - January 26th}
To Do:
\begin{itemize}
\item Complete simuTrans integrator\\
\item No surface brightness testing\\
\item Limb darkening model ( both quadratic and square root)\\
\item Von Zipel gravity darkening model\\
\item More complicated gravity darkening model\\
\item Model on real data (starting with LD models)\\
\item MCMC python wrapper\\
\item Test on simulated data \& some real data\\
\end{document}
